In the regression problem, each of the training samples $f_i$ have a
response $Y_i$ associated with them. This response can be either scalar or
functional, although the way to proceed will be similar.

In both cases, it is necessary to select the responses associated with the
elements of the neighborhood $k(x)$, which will be used to predict the response
of the datum $x$.

In the scalar response case, a weighted average of the neighbors’ responses
$Y_i$ is used, so that the prediction will be calculated as

\begin{equation}{Regression response}
\hat Y = \sum_{(f_i, Y_i) : f_i \in k(x)} w_i Y_i \, ,
\end{equation}

where $\sum w_i = 1$.
which may be chosen based on distance or uniformly.

In the case where the responses are also functional data,
the predicted response is constructed in a similar way as in the previous
case, using a weighted average of functions, or a centroid, such as the
Karcher means presented in Section \ref{SEC:ELASTICREG}.

During this chapter we have focused on making a brief description of the
mathematical concepts used in this work. The aim of this work was to include
all these concepts as functionalities in the \textit{scikit-fda} package.
The next chapter summarizes the design decisions followed for
this task, the technologies used for it and the methodologies used for this purpose.
