In the regression problem, each of the training samples $\mathcal{X}_i$ have a
response $Y_i$ associated with them. This response can be either scalar or
functional, although the way to proceed will be similar.

In both cases, it will be necessary to select the responses associated with the
elements of the neighborhood $k(x)$, which will be used to predict the response
of the datum $x$.

In the scalar response case, a weighted average of the neighbors’ responses
$Y_i$ will be used, so that  the prediction will be calculated as

\begin{equation}{Regression response}
\hat Y = \sum_{(X_i, Y_i) : X_i \in k(x)} w_i Y_i \, ,
\end{equation}

where $\sum w_i = 1$.
which may be chosen based on distance or uniformly.

In the case where the responses $\mathcal{Y}_i$ are also functional data,
the predicted response will be constructed in a similar way as in the previous
case, using a weighted average of functions, or a centroid, such as the
Karcher means presented in section \ref{SEC:ELASTICREG}.
