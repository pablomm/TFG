
In the preprocessing module a specific module called \textit{registration} has
been created to deal with the registration of the data using the techniques
explained throughout the sections \ref{SEC:REGISTRATION} and \ref{SEC:ELASTIC}.
Among the functions incorporated in this module are \textit{shift\_registration},
for registrations using shifts,\\ \textit{landmark\_registration},
to employ known landmarks, or \textit{elastic\_registration} with the algorithm
detailed in Section \ref{SEC:ELASTICREG}.
In addition, this module contains functions for the treatment of
warpings functions among other useful processes during the registration phase.
To ensure the efficiency of the algorithm used in the pairwise alignment
(eqn. \ref{EQ:ENERGY}), it was implemented using \textit{C},
which is called by the Python package through \textit{Cython}.

All the registration methods have been designed to receive an \textit{FData} object and
return another one containing the registered samples, which can be used with
other functionalities of the package, such as the classification or regression
models included in the machine learning module.

%In the preprocessing module a specific sub-module was created to deal with the
%registration problem, with methods to deal with the data with the techniques
%studied throughout sections \ref{SEC:REGISTRATION} and \ref{SEC:ELASTIC},
%allowing data to be registered by means of shifts, landmarks or the elastic
%approach.
