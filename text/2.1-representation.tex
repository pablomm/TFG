Since our observations will be functions, dependent on one or several continuous
parameters, such as the time in which a process takes place, it will be
impossible to measure and store all the points which they are defined for.

In practice, the information of a functional datum $x(t)$ is observed and
recorded as pairs $x(t_j) = (t_j, y_j)$, where $y_j$ is a snatshoot of the
function $x$ at $t_j$, possibly blurred by measurement error\cite{Ramsay2005}.

Although in FDA is also treated the case in which our data are multivariate
functions, from  $\mathbb{R}^n$ to $\mathbb{R}^m$, in general, as a way of
simplification, we will assume that our data are univariate functions.

There are two main approaches to represent the data. The first, called discrete
representation, stores a finite grid of pairs $(t_j, y_j)$ with the recorded
values, but unlike multivariate statistics, the continuous dependence between
$y_j$ and $t_j$ will be taken into account.

The second one is a parametric approach, called basis representation, in which
we will define a system of functions $\{\phi_n(t) \}_{n=1}^N$, such as a
truncated Fourier basis or a set of polynomials. This system will form a finite
subspace of the original functional space, where we will project the
observations, obtaining a representation associated to the coefficients of
the basis $x(t) \approx \sum_{n=1}^N c_n\phi_n(t)$.
