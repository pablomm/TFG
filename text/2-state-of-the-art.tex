In FDA, the analyzed data might take the form of temporal series, curves,
surfaces or anything else varying over a continuum.  The observations that will
make up our data will come from random variables, whose realizations will take
values in functional spaces, that is to say, they will be realizations of
stochastic processes. For this reason, under this framework, we will consider
them functions.

One may wonder what makes FDA different from multivariate analysis, considering
the fact that data is generally collected and stored in the form of discrete
point sets ${(t_j, y_j)}$ \cite{Srivastava2016}.  In
multivariate statistics one works with the vector ${y_j}$, applying statistical
methods and vector calculus to its study. However, FDA keeps the association of
values ${y_j}$ and ${t_j}$, taking into account the functional nature of the
data, allowing the study of the data using functional calculus.

In this chapter we will make an overview, from a theoretical point of view, of
different FDA topics covered in this work.
