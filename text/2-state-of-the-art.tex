In FDA, the analyzed data can take the form of temporal series, curves,
surfaces or any process that varies over a continuum. The observations that
make up our data come from random variables, whose realizations take
values in functional spaces, that is to say, they are realizations of
stochastic processes. For this reason, they can be viewed as random functions.

In practice, the information of a functional datum $f(t)$ is observed and
recorded as pairs $(t_j, y_j)$, where $y_j$ is a snatshoot of the
function $f$ at $t_j$, i.e., $f(t_j)$ and $t_1 < t_2 < \dots$.
One may wonder what makes FDA different from multivariate analysis, considering
the fact that data is generally recorded and stored in the form of
observations at discrete points \cite{Srivastava2016}.
In multivariate statistics one works with the vector of values $\{y_j\}$,
applying statistical methods and vector calculus to its study,
But without taking into account the high correlation between the values
$y_j, and $y_{j+1}$, due to the continuous structure given by the parameter on
which they vary. The order of the values $t_j$ is fundamental  due to its
functional nature.

However, FDA keeps the association of
values ${y_j}$ and ${t_j}$, taking into account the functional nature of the
data, allowing the study of the data using functional calculus.
Since the observations are functions, dependent on one or several continuous
parameters, such as the time in which a process takes place, it is not
possible to measure and store all the points which they are defined for.

There are two main approaches in FDA to represent the data while preserving its
functional structure. The first, called \textit{discrete representation}, stores
a finite grid of pairs $(t_j, y_j)$ with the recorded
values, but unlike multivariate statistics, the continuous dependence between
$y_j$ and $t_j$ is taken into account. These pairs represent the values
$f(t_j) = y_j$ of a functional datum $f$. Interpolation is used o reconstruct
the data as a continuous function from its discrete values.

The second one is a parametric approach, called \textit{basis representation},
in which is considered a system of functions $\{\phi_k(t) \}_{k=1}^K$, such as a
truncated Fourier basis or a set of polynomials. This system forms a finite
subspace of the original functional space, where the
observations are projected, obtaining a representation associated to the
coefficients of the basis $f(t) \approx \sum_{k=1}^K c_k\phi_k(t)$.

Due to this functional structure in the data,
FDA needs to address some issues that do not arise when studying other types of
data, such as vectors.
For instance, once our data is represented as a set of functions
$\{f_i(t)\}_{i=1}^{n}$, its variability can proceed from two sources,
the random curve-to-curve variation, i.e., the difference between the values of
the functions, or from its internal temporal structure.
It is fundamental to study and quantify these two types of variability in
the analysis of our data. This separation is called the registration of the data,
and is treated in detail in the Section \ref{SEC:REGISTRATION}.

FDA also deals with topics present in classical statistics in cases where the
data are functions. Among these problems are the study of regression and
classification models. In the section \ref{SEC:NEIGHBORS} the use of nearest
neighbors estimators applied to functional data is studied.

Although in FDA is also treated the case in which our data are multivariate
functions, from  $\mathbb{R}^d$ to $\mathbb{R}^m$, in general, as a way of
simplification, during this chapter we will assume that our data are
univariate functions.

In this chapter is made an overview, from a theoretical point of view, of
different FDA topics covered in this work. However, this chapter also serves
to introduce the functionalities included in the \textit{scikit-fda} package.
The figures shown throughout this section have been made using these functionalities.
