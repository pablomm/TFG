
In the discrete representation, frequently, it will be necessary to resample or
evaluate the data, or its derivatives, at points within the domain range,
different that the original pairs $\{(t_j, y_j)\}$ at which our observations
have been measured. An example of this it is shown in the figure
\ref{FIG:RESAMPLE}. For this purpose, we will use interpolation.

\begin{figure}[Function resampled using interpolation]{FIG:RESAMPLE}
  {Function resampled using interpolation}
  
  \subfigure[SBFIG:RESAMPLE1]{Original observation}{\image{7cm}{}{0-empty}} \quad
  \subfigure[SBFIG:RESAMPLE2]{Observation resampled}{\image{7cm}{}{0-empty}}
\end{figure}

Although they are not the only methods used for interpolation, splines and
smoothing splines are the most used, for this reason we will focus on them
during this work.
