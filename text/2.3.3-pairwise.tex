
The Fisher-Rao metric will allow us to formulate a suitable criterion for the
pairwise alignment problem, as presented in section \ref{SRC:L2PAIRWISE},
 that avoids all the problems associated with the metric $\mathbb{L}^2$.

Given $f_1, f_2 \in \mathscr{F}$, to register f_1 to f_2 the Fisher-Rao distance
will be minimized, as energy term, i.e., the warping used in the alignment will be

$$
\gamma^{*}= \operatorname{argmin}_{\gamma \in \Gamma} d_{FR}[f_1 \circ \gamma,
f_2] = \operatorname{argmin}_{\gamma \in \Gamma} \|
(q_1, \gamma) – q_2 \|_{\mathbb{L}^2}.
$$

We will have a symmetry property, since
$E[f_1, f_2] = E[f_1 \circ \gamma, f_2 \circ \gamma]$, the warping used to
register $f_1 \circ \gamma$ to $f_2 \circ \gamma$ will be the same as in the
previous case.

Because of this property, our problem will have inverse consistency, since
$E[f_1 \circ \gamma, f_2] = E[f_1 \circ \gamma \gamma \gamma^{-1}, f_2
\circ \gamma^{-1}] = E[f_1, f_2 \circ \gamma^{-1}]$, that implies that the
warping needed to register $f_2$ to $f_1$  is $\gamma^*^{-1}$, as it is shown in
\ref{FIG:INVERSE}.

\begin{figure}[Inverse consistency of pairwise alignment]{FIG:INVERSE}{Inverse consistency of pairwise alignment}
  \subfigure[SBFIG:INVERSE1]{Alignment of $f_1$ to $f_2$}{\image{7cm}{}{0-empty}} \quad
  \subfigure[SBFIG:INVERSE1]{Alignment of $f_2$ to $f_1s$}{\image{7cm}{}{0-empty}}
\end{figure}

In addition, the so-called pinching effect will not appear, due to the term
$\sqrt{\dot \gamma}$ that is present in the criterion to minimize.
