In descriptive statistics, given a set of random points
$\{x_i\}_{i=1}^{N} \subset \mathbb{R}^n$, the sample mean
$\bar x = \frac{1}{N}\sum_{i=1}^{N} x_i$ is used to estimate the central
tendency of the data. This mean minimizes the sum of square distances.

In the functional case, the mean function will have this property thus if
$\{f_i\}_{i=1}^{N} \subset \mathbb{L}^2$ is a set of functional observations,
their mean $\bar f(t) = \frac{1}{N}\sum_{i=1}^{N} f_i(t)$ will minimize
$\sum_{i=1}^{N}\|f_i - \bar f\|_{\mathbb{L}^2}$.

We are interested in extend this idea to general metrics spaces.
Let $(X, d)$ be a metric space and $\{x_i\}_{i=1}^{N}$ random points in $X$.
The Fréchet variance of a point $p \in X$ as $\Psi(p)=\sum_{i=1}^{N} d^{2}\left(p, x_{i}\right)$.

The Fréchet mean of these random points will be defined as the element $m \in X$
which globally minimizes the Fréchet Variance. If this global minimum does not
exist, we will call Karcher means to the points that locally minimizes
$\Psi(p)$, i. e.,

$$
m=\underset{p \in M}{\arg \min } \sum_{i=1}^{N} d^{2}\left(p, x_{\dot{i}}\right)
$$

These Karcher means will be able to better capture the geometry of the problem
than usual mean, as it is shown in the figure \ref{SBFIG:KARCHER2}, where it is
used a Karcher mean on $\mathscr{A}$.
To calculate this mean it is used the DPA algorithm explained in the appendix B.2.
For instance, the dataset shown in \ref{SBFIG:KARCHER1} contains gaussian-like
samples, however the usual mean of figure \ref{SBFIG:KARCHER2} is not able to reflect
this shape, unlike the Karcher mean in $\mathscr{A}$.

\begin{figure}[Karcher mean of dataset]{FIG:KARCHER}{Karcher mean of dataset}
  \subfigure[SBFIG:KARCHER1]{Dataset of unimodal samples}{\image{7cm}{}{unimodal-dataset}} \quad
  \subfigure[SBFIG:KARCHER2]{Usual mean and Karcher mean on $\mathscr{A}$}{\image{7cm}{}{means}} \\
  \subfigure[SBFIG:RANK1]{Ranked by elastic distance}{\image{7cm}{}{dataset-by-amplitude}} \quad
  \subfigure[SBFIG:RANK2]{Ranked by phase distance}{\image{7cm}{}{dataset-by-phase}}
\end{figure}

As an example of the behavior of these means with different metrics, we may
measure the centrality of an observation in a dataset using its distance to the
Karcher mean. In the figure \ref{KARCHER} it is used this idea to rank a dataset of
unimodal samples. Reddish colors indicate higher centrality of a sample, i.e.,
a smaller distance to the mean, and lighter colors indicates outlier samples.
In the figure \ref{SBFIG:RANK1} it is used the amplitude distance, and their
corresponding Karcher mean. We can observe that the location of the mode in a
sample does not affect its centrality, in contrast to the phase distance,
as it is shown in the figure \ref{SBFIG:RANK2}.
