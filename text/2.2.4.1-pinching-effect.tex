


 \begin{figure}[Pinching force effect]{FIG:PINCHING}{Pinching force effect}
   \subfigure[SBFIG:PINCHING1]{Alignment of $f_1$ to $f_2$}{\image{7cm}{}{0-empty}} \quad
   \subfigure[SBFIG:PINCHING2]{Pinching effect}{\image{7cm}{}{0-empty}}
 \end{figure}

It is a phenomenon we may encounter when we try to align two functions without a
perfect match. To minimize the term energy, the optimal solution will tend to
squeeze an annoying region, until it disappears. Figure \ref{FIG:PINCHING}
shows the result of the registration of two functions with pinching effect.