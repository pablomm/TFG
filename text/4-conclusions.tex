
During this year, the project has evolved drastically, redesigning and expanding
its functionalities.
As a result of this work, the \textit{scikit-fda} package incorporates
interpolation techniques, which allow the evaluation of functional data as
continuous functions. In addition, a wide variety of data registration
techniques and elastic methods useful in \acs{FDA} have been included.
Through work with nearest neighbors estimators,
the package provides models to address the classification and regression problems
with functional data.

Among the aspects that have allowed this development to be successfully achieved
are communication between team members, with weekly meetings, and the use of
an CI system which has made it much easier to integrate working together.
In addition, the experience of the team throughout the last year in which the
project was initiated has been fundamental.

In spite of all the advances made during this year there is still a lot of work
to be done.
It would be interesting to expand the basis
representations of functional data, including support for surfaces, so that
covariances can be represented in this form, and provide support to more types of basis,
such as wavelets \cite{Morettin2017} or constrained basis\cite{Ramsay2005}.

It would also be interesting to extend the registration techniques,
incorporating functionalities from the \textit{fdasrvf}\cite{fdasrvf} package,
which include tools for clustering and registration, or for elastic
alignment of spatial curves and surfaces.

Leaving aside these technical aspects, one of the most important tasks to carry
out is to give visibility to the project, discovering the library to interested
developers and researchers, presenting the package in congresses and publishing
articles to promote its use.

To conclude, I would like to make a brief summary of the skills acquired during
the course of this work. On the one hand, this work has allowed me to learn concepts
of functional data analysis, as well as knowledge of scientific programming
and the requirements and design that software of this type must meet.
On the other hand, due to the collaboration together with a team,
I have been able to learn dynamics of work for the development of software,
and for the collaboration in team projects.
All this  makes this work personally worthwhile,  and has served to acquire a
global perspective of the knowledge acquired in this degree.
