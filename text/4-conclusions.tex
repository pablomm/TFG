The goal of this project is the creation of a Python reference package in
FDA, along with the creation of a community to
support and keep the project updated. So it will require the effort and
dedication of a team of developers to achieve this goal.

During this year, the project has evolved drastically, redesigning and expanding
its functionalities.
Among the aspects that have allowed this development to be successfully achieved
are communication between team members, with weekly meetings, and the use of
an CI system which has made it much easier to integrate working together.
In addition, the experience of the team throughout the last year in which the
project was initiated has been fundamental.

In spite of all the advances made during this year there is still a lot of work
to be done.
Among these functionalities that it would be interesting to cover the basis
representations of functional data, including support for surfaces, so that
covariances can be represented in this form, and to support more types of basis,
such as wavelets \cite{Morettin2017} or constrained basis\cite{Ramsay2005}.

It would also be interesting to extend registration techniques,
incorporating functionalitiesfrom the \textit{fdasrvf}\cite{fdasrvf} package,
which include tools for clustering and registration, or for elastic
alignment of spatial curves and surfaces.

Leaving aside these technical aspects, one of the most important tasks to carry
out is to give visibility to the project, to discover the library to interested
developers and researchers, presenting the package in congresses and publishing
articles to promote its use.
