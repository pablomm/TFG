
Sometimes it is necessary to control the amount of elasticity during the
registration process, for this purpose it is possible to add a penalty term
$\mathcal{R}(\lambda)$ to the elastic distance.
Let $q_1, q_2 \in \mathbb{L}^2$ be two SRSFs and $\lambda > 0$, we will define
the penalized elastic distance as

$$
d_{\lambda}\left(q_{1}, q_{2}\right) \equiv \inf _{\gamma \in \Gamma}\left(
\| q_{1}-\left(q_{2} \circ \gamma \sqrt{\dot{\gamma}}\right)\left\|^{2}+
\lambda\right\| \sqrt{\dot{\gamma}}-1 \|^{2} \right)^{(1 / 2)}.
$$

The alignment of two functions with different penalties is shown in the figure \ref{FIG:PENALTY}.

\begin{figure}[Penalized elastic registration]{FIG:PENALTY}{Penalized elastic registration}
  \image{7cm}{}{0-empty}
\end{figure}

\begin{figure}[Penalized elastic registration]{FIG:AMPPHA}{Penalized elastic registration}
  \subfigure[SBFIG:AMPLITUDE]{Registration of $f_1$ to $f_2$ with a penalty term}{\image{7cm}{}{penalty-elastic}} \quad
  \subfigure[SBFIG:PHASE]{Warping used in the alignment}{\image{7cm}{}{penalty-elastic-warping}}
\end{figure}
