

For the calculation of the values $\delta_i$ necessary for the shift registration of a set  $\{x_i\}_{i=1}^{n}$, according to \ref{EQN:SHIFTS},
we will use a variation of the Newton-Rhapson's root-finding algorithm, applied
to the derivative of REGSSE (\ref{EQ:REGSSE}). This procedure is explained in more detail
in Ramsay and Silverman (2005) \cite{Ramsay2005}.

For this computation it will be necessary to evaluate derivatives of
$x_i$, so it will be crucial a previous smoothing step of the samples.
Derivatives of REGGSE are given by:

\begin{equation}[EQ:FIRST]{First derivative of REGSSE}
\frac{\partial}{\partial \delta_i} \text{REGSSE} = 2 \int_{\mathcal{T}}
\left [ x_i(t + \delta_i) - \hat \mu(t) \right ] Dx_i(t) dt,
\end{equation}

\begin{equation}[EQ:SECOND]{Second derivative of REGSSE}
\frac{\partial^2}{\partial \delta^2_i} \text{REGSSE} = 2
\int_{\mathcal{T}} \left [ x_i(t + \delta_i) - \hat \mu(t) \right ]
D^2x_i(t) dt + 2 \int_{\mathcal{T}} \left [  Dx_i(t)^2 \right ]  dt.
\end{equation}

In practice the first term of $\frac{\partial^2}{\partial \delta^2_i}\textbf{REGSSE}$
(\ref{EQ:SECOND}) it is deleted, because when the misalignment of the samples is large it
can affect the convergence of the algorithm, and vanishes for the values that
minimize the criterion. Therefore the following approximation is used:

\begin{equation}[]{Aproximation of second derivative of REGSSE}
\frac{\partial^2}{\partial \delta^2_i} \text{REGSSE} \approx  
2 \int_{\mathcal{T}} \left [  Dx_i(t)^2 \right ]  dt.
\end{equation}

\begin{algorithmN}[ALG:NEWTON]{Shift registration by Rhapson-Newton algorithm}{Shift registration by Rhapson-Newton algorithm}
	\SetKwData{delta0}{$\delta{(0)}$}
	\SetKwData{This}{this}
	\SetKwData{Up}{up}
	\SetKwFunction{Union}{Union}
	\SetKwFunction{FindCompress}{FindCompress}
	\SetKwInOut{Input}{Input}
	\SetKwInOut{Output}{Output}
	\Input{Set of functional observations $\{x_i(t)\}_{i=1}^{n}$}
	\Output{Shifts $\{\delta_i\}_{i=1}^{n}$ used to register the data}
	\BlankLine
	Initialize $\delta^{(0)}$ \;
	
	\For{step $\nu = 1, 2, \dots$ until stop criterion}{
		\emph{Update cross-sectional mean}
		
		$\hat \mu(t) \leftarrow \frac{1}{n} \sum_{i=1}^{n} x_i(t + \delta_i^{(\nu -1)})$
		
		\lForEach{$\delta_i ^{(\nu - 1)}$}{
			
			\emph{Update values of $\delta_i$}
			
			$\delta_i ^{(\nu)} \leftarrow \delta_i ^{(\nu - 1)} - \alpha {\frac{\partial}{\partial \delta_i}  REGSSE} / {\frac{\partial^2}{\partial \delta^2_i} REGSSE}$
		}
	}
\end{algorithmN}

%Step 0:
%
%Initialize  \delta_i ^{(0)}
%
%Step \nu: for \nu=1,2,3,…
%
%\hat \mu   <- \frac{1}{N} \sum x_i(t + \delta_i^{(\nu -1)})
%\delta_i ^{(\nu)} = \delta_i ^{(\nu - 1)} - \alpha \frac{\frac{\partial}{\partial \delta_i}  REGSSE}{\frac{\partial^2}{\partial \delta^2_i} REGSSE}


The initilization of $\delta_i ^{(0)}$  in \ref{ALG:NEWTON} may be set to minimize some feature, or
simply set $\delta_i ^{(0)}$ to $0$.

Could be used as stop criterion a maximun value of iterations along with a
tolerance \\ $| \delta_i ^{(\nu)} - \delta_i ^{(\nu-1)} | < \epsilon$.
Generally the convergence is fast, obtaining good alignments with one or two
iteration with a reasonable estimation of the initial values $\delta_i ^{(0)}$. The step size
$\alpha$ in \ref{ALG:NEWTON} may be set to $1$.
