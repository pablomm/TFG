Let  $f_1, f_2 \in \mathcal{F}$ be two functions aligned. Generally, if we apply
a reparameterization $\gamma\in\Gamma$ to both functions, their distance will
change, thus

$$
\| f_1 \circ \gamma - f_2 \circ \gamma \|^2 = \int_\mathcal{T} (f_1(\gamma(t)) -
f_2(\gamma(t))|^2dt =
\int_\mathcal{T} |f_1(s) - f_2(s)|^2 \frac{1}{\dot \gamma ( \gamma^{-1}(s))} ds \qquad
(s=\gamma(t)) \, ,
$$

in other words, the action of $\Gamma$ over $\mathcal{F}$ under de $\mathbb{L}^2$
metric is not by isometries.

Therefore $E[f_1, f_2] \neq E[f_1 \circ \gamma, f_2 \circ \gamma]$ and
consequently $f_1 \circ \gamma$ y$ f_2 \circ \gamma$ may not be aligned.
A direct consequence of this issue will be the inverse inconsistency of the
problem, because of $E[f_1 \circ \gamma, f_2] \neq
E[f_1, f_2 \circ \gamma^{-1}]$. It would be desirable that the transformation
necessary to align $f_1$ to $f_2$ be the inverse transformation to align $f_2$
to $f_1$.
The search of a more adequate energy for this problem will motivate the section
\ref{SEC:ELASTIC} of this work.
