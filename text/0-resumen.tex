El análisis de datos funcionales (FDA) es una rama de la estadística dedicada
al estudio de cantidades aleatorias que dependen de un parámetro continuo,
como series temporales o curvas en el espacio. En FDA, los datos pueden ser
vistos como funciones aleatorias muestreadas de un proceso
estocástico subyacente.

Es este trabajo consideraremos tres tareas diferentes en FDA:
el uso de técnicas de interpolación para estimar valores de las funciones
en puntos no observados,
el registro de este tipo de datos
y los problemas de clasificación y regresión cuando las instancias
son caracterizadas por atributos funcionales.
En particular, en este proyecto se han extendido las funcionalidades del
paquete de Python \textit{scikit-fda} para dar soporte a estas tres áreas.

En general, las instancias de datos consideradas en FDA están formadas por una
colección de observaciones medidas en valores discretos del parámetro del
que dependen (p. ej. tiempo o espacio).
Para algunas aplicaciones es conveniente, y en algunos casos necesario,
estimar el valor de estas funciones en puntos no observados.
Esto puede lograrse mediante el uso de interpolación a partir de las
mediciones disponibles.

En algunas aplicaciones, las funciones observadas tiene formas similares,
pero presentan una variabilidad cuyo origen proviene de
distorsiones en la escala del parámetro continuo del que
dependen. El registro de los datos consiste en caracterizar esta variabilidad
y eliminarla de la muestra.


En este trabajo también se abordan los problemas de clasificación y regresión
con datos de naturaleza funcional. Concretamente, se han diseñado
estimadores de vecinos próximos, basados en la noción de cercanía entre muestras.

En particular, en este trabajo se ha extendido el paquete \textit{scikit-fda}
para incluir métodos de interpolación basados en splines. También se ha dotado
el paquete con herramientas para el registro de datos usando traslaciones,
puntos de referencia o para el registro elástico, el cual hace uso
de la métrica de Fisher-Rao para alinear las funciones de una muestra.
Además, se han incluido modelos basados en vecinos próximos para realizar
regresión, tanto con respuesta escalar como funcional, y clasificación.
