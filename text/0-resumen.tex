El análisis de datos funcionales, también denominado FDA, es una rama de la
estadística encargada del estudio de variables aleatorias de naturaleza
funcional, como puede ser un conjunto de series temporales o de curvas en el
espacio. Por esta razón en FDA los datos tratados son considerados funciones.

Durante este trabajo se han tratado tres partes fundamentales de este campo:
la representación de los datos como funciones, el estudio de su variabilidad
y los problemas de clasificación y regresión.

Generalmente este tipo de datos son observados y medidos en un conjunto
discreto de puntos, por lo que será necesaria su representación como
funciones continuas a partir de estas mediciones.
Uno de los enfoques empleados, para ello, en el cual nos centraremos, es el uso de interpolación.
Esta representación funcional resultará fundamental durante el análisis posterior,
como en el estudio de la variabilidad de un conjunto de muestras.

Debido a su naturaleza continua, la variabilidad de los datos
puede proceder de su escala interna, como el tiempo en el que evoluciona un
proceso. La fase en la que este tipo variabilidad es cuantificada y separada se
conoce como el registro de los datos. Tras esta etapa de preproceso, es posible
abordar otros temas clásicos en estadística, como son los problemas de
clasificación y regresión. Entre los modelos usados para ello se encuentran los
estimadores de vecinos próximos, basados en la noción de cercanía entre los
datos de una muestra, los cuales serán tratados en este trabajo.

En el año 2017 surgió el proyecto \textit{scikit-fda}, con el objetivo de crear
un paquete en Python que diera soporte a este estudio. El objetivo principal
de este trabajo ha sido dotar al paquete de funcionalidades relativas a la
representación, registro, clasificación y regresión de datos funcionales.


%En las últimas dos décadas las técnicas empleadas para este análisis han
%evolucionado rápidamente, así como sus aplicaciones en gran cantidad de campos
%como medicina, bioinformática o ingeniería.
%Esta técnica permite la construcción de funciones a partir de un conjunto de
%mediciones discretas de los datos.
