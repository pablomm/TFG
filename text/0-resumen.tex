El análisis de datos funcionales, también denominado FDA, es una rama de la 
estadística encargada del estudio de variables aleatorias de naturaleza
funcional, como puede ser un conjunto de series temporales o de curvas en el
espacio.

En las últimas dos décadas las técnicas empleadas para este análisis han
evolucionado rápidamente, así como sus aplicaciones en gran cantidad de campos
como medicina, bioinformática o ingeniería.

Sin embargo, debido a la relativamente reciente aparición de este campo, no hay
una gran diversidad de soluciones software que engloben estas técnicas. Por esta
razón en el año 2017 surgió el proyecto open-source denominado scikit-fda, con
el objetivo de crear un paquete en Python que diera soporte a este estudio,
junto con una comunidad que contribuyera a su desarrollo.

Con este trabajo se pretende contribuir al crecimiento de este proyecto, cuyo
objetivo a largo plazo es convertirse en una referencia en FDA que contenga una
amplia variedad de herramientas para el estudio de estos datos.

En concreto, durante este trabajo se han realizado contribuciones en diferentes
áreas de este campo, entre las que se encuentran el registro de los datos, etapa
en la cual se estudia la variabilidad debida a su escala interna, como puede
ser el tiempo en el que evoluciona un proceso; o su representación, en la cual
se estudian los problemas asociados su tratamiento y almacenamiento.
