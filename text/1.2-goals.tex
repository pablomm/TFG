The aim of this work is to continue the \textit{scikit-fda} project,
collaborating with in a general way in its development, but especially in the
registration, one of the areas of FDA, in which it is studied the
variability of the data due to its internal scale, such as time
in which a process unfolds. Due to the extent of this area, it is not possible
to cover all techniques in this field, therefore the purpose of this work was
not delimited from the beginning in a clear way.

For this reason it was decided to start with the techniques
exposed in Functional data analysis[*], which makes an general overview of different
aspects of FDA. After this it was developed some
novel tools in the alignment of functional data, which are described in detail in the
section \ref{SEC:ELASTIC}. To finalize, there were implemented models based on
nearest neighbors for regression and classification.

Throughout the year, weekly meetings were scheduled with all the team involved
in the project, in which the status of the various functionalities was reviewed
and the next steps in development were evaluated.
