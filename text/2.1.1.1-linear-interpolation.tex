
Given two points $t_k$ and $t_{k+1}$ for which the values of our function are
known, denoted by $y_k$ and $y_{k+1}$, we can use a line segment to join these
values as a first approach. Using this type interpolation, we will obtain a
piecewise linear function, whose values in the interval $[t_k, t_{k+1}]$ will
be given by

\begin{equation}[]{Linear interpolation}
 x(t)= y_{k}+\frac{y_{k+1}-y_{k}}{t_{k+1}-t_{k}}\left(t-t_{k}\right).
\end{equation}

In the case of multivariate functions, multilinear interpolation
[ref to multilinear interpolation] generalizes this method to higher dimensions.
The same principles will be applied, but the piecewise domain will be composed
by regions calculated using triangulation, and the evaluation will be performed
using a multilinear form, obtaining in the case of surfaces piecewise
functions formed by plane sections. In the figure \ref{FIG:LINEAR} it is plotted
the result of interpolating a temporal series and a surface.


\begin{figure}[Example of linear interpolation]{FIG:LINEAR}{Example of linear interpolation}
  \subfigure[SBFIG:LINEAR1]{Linear interpolation}{\image{7cm}{}{linear-interpolation}} \quad
  \subfigure[SBFIG:LINEAR2]{Bilinear interpolation}{\image{7cm}{}{surface-bilinear}}
\end{figure}
