

Functional data analysis (FDA), is a branch of the
statistic which deals with the study of random variables of
functional nature, such as time series or curves in the
space. It is a relatively recent field, whose first references
began in the 1950s, with various articles related to the study of stochastic
processes. However, it was not until 1982 with the publication by
J.O. Ramsay of \textit{When the data are functions}[*] when the term FDA began
to be used to denote so this field.

Since then, and especially in the last two decades, techniques used for this analysis
have evolved quickly, as well as their applications in a wide range of fields
such as medicine, bioinformatics or engineering.

Although there were some software solutions to support this field, such as
\textit{fda}[*], powered by J.O. Ramsay; or \textit{fda.usc}[*], powered by the
University of Santiago de Compostela. But none of them had been implemented
under the philosophy of a free software project, encouraging open collaboration.
In addition all implementations were based on R[] or Matlab[], and there were no
Python packages for this purpose, although its use has become widespread in
recent years in statistics and machine learning.

In 2017, the \textit{scikit-fda} project came into being,
under the name \textit{fda}[*], as part of a degree work made by Miguel Carbajo
at the UAM. The aim of this work was to initiate the creation of a
Python package to give support to this field, under the philosophy of a open-source
project, along with the creation of a community that contributes to development
and maintenance. Currently the project is being driven by the machine
learning group of the UAM (GAA-UAM) together with the contributions of several
degree works, including this one.

Although the project is in an early stage of development, the package is already
in use by some researchers and was presented to the public in the
III International Workshop on Advances in Functional Data Analysis[*].
