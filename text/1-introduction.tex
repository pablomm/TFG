\newacronym{FDA}{FDA}{Functional data analysis}

\ac{FDA}, is a branch of
Statistics that deals with the study of random variables of
functional nature, such as time series or curves in the
space. It is a relatively recent field, whose first references
began in the 1950s, with various articles related to the study of stochastic
processes. However, it was not until 1982 with the publication by
J.O. Ramsay of \textit{When the data are functions}\cite{Ramsay1982} when the
term \acs{FDA} began to be used to denote this field.
Since then, and especially in the last two decades, techniques used for this analysis
have evolved quickly, as well as their applications in a wide range of fields
such as medicine, bioinformatics or engineering.

Due to their continuous structure, the data treated in \acs{FDA} are viewed as functions.
However, functional data are generally observed and recorded as a discrete set of
measures. For this reason, one of the first tasks to be performed in \acs{FDA} is the
representation of these data as continuous functions.
One of the main approaches to address this task is the use of interpolation
for the construction of continuous functions from these discrete measurements.

Once the data are in functional form, it is possible to perform
further analysis, such as studying the variability of a set of samples.
Unlike other types of data, due to their functional nature, the variability of
a set of functional samples may come from their internal structure, such as the time scale
in which a process takes place. For this reason, it is necessary to
incorporate a stage in the analysis in which this variability is quantified
and separated. This step is called registration.

Throughout this work we will focus primarily on three areas of \acs{FDA}:
the use of interpolation for data representation, the registration, and the use
of nearest neighbors estimators on classification and regression problems with
functional data.

There are some software solutions in R and Matlab that provide support to this
field, such as \textit{fda}\cite{fda-r}\cite{Ramsay2009};
or \textit{fda.usc}\cite{FdaUsc}, developed by a team at the
University of Santiago de Compostela.
Inspired by these solutions, in 2017, the \textit{scikit-fda} project arose
under the name \textit{fda}\cite{FDA2018}, as part of a Bachelor's degree thesis
made by Miguel Carbajo at the UAM. The aim of this work was to initiate the creation of a
Python package to give support to this field, under the philosophy of an open-source
project, along with the creation of a community that contributes to its development
and maintenance.

Currently, the project is being driven by the Machine
Learning Group at the UAM together with the contributions of several
undergraduate thesis, such as this one.
The package is already in use by some researchers and was presented
in the \textit{III International Workshop on Advances in Functional Data
Analysis}\cite{ramos-carrenoScikitfdaPythonPackage2019}.
