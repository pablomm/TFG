Functional data analysis, also called FDA, is a branch of statistics that deals
with the study of random variables of a functional nature, such as temporal
series or curves in space.

In the last two decades, techniques used for this analysis have evolved quickly,
as well as their applications in many fields such as medicine, bioinformatics
or engineering.

However, due to the relatively recent appearance of this field, there is not a
great variety of software solutions that encompass these techniques. For this
reason, in 2017 the open-source project called scikit-fda arose, with the aim of
creating a Python package which would support this study, along with a
community that would contribute to its development.

This work aims to contribute to the growth of this project, whose long-term goal
is to become a reference in FDA, containing a wide variety of tools for the
study of these data.

Specifically, during this work contributions have been made in different areas
of this field, including the registration of data, a stage in which the
variability due to its internal scale is studied, such as the time in which a
process evolves; or its representation, in which the problems associated with
its treatment and storage are studied.
