\textcolor{red}{TODO: Revisar redacción y erratas. Glosario de términos. Ejemplo parte 3.}

Functional data analysis, also called FDA, is a branch of Statistics that deals
with the study of random variables of a functional nature, such as temporal
series or curves in space. For this reason in FDA the data treated are
considered functions.

During this work we focus on three fundamental parts of this field:
the representation of data as functions, the study of their variability,
and the classification and regression problems.

Generally, this type of data is observed and measured in a discrete set of
points, so it is necessary to represent them as continuous functions using
these measurements. One of the approaches employed, on which we will focus, is the
use of interpolation. This functional representation will be fundamental during
the subsequent analysis, as in the study of the variability of a set of samples.

Because of its continuous nature, the data variability may be due to its
internal scale, such as the time in which a process evolves.
The part in which this type of variability is quantified and separated is known
as registration.

Due to its continuous nature, data variability can come from its internal scale,
such as the time in which a process evolves. The part in which this
type of variability is quantified and separated is known as registration.
After this pre-processing step, it is possible to address other classic
issues in Statistics, such as the problems of classification and regression.
Among the models used for this are the estimators of nearest neighbors,
based on the notion of closeness among samples. These estimators
will be treated in this work.

In 2017, the project \textit{scikit-fda} arose, with the aim of creating a
Python package to support this study. The main purpose of this work has been to
provide the package with functionalities relating to the representation,
registration, classification and regression of functional data.
