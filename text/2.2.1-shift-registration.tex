A first approach to solve this problem was presented in Ramsay and Silverman
(2005) [*], by considering a simple shift in the domain to make the
registration. Although it is a basic approach, it will be useful in many cases.
Figure \ref{SBFIG:SHIFT1} shows a set of sinusoidal waves whose phases are not
aligned, and a shift in the time scale will be adequate to make the alignment.

Let $\{x_i\}_{i=1}^n$ be a set of functional observations, which will be aligned
using this transformation. We are actually interested in finding the values

$$
x_i^*(t)=x_i(t+ \sigma_i) \, i=1,2, \dots n
$$

where the shift parameter $\sigma_i$ is chosen in order to align the features of
the curves. In the figure \ref{SBFIG:SHIFT2} it is shown the result of apply
this type of transformation to the set of sinusoidal waves.

\begin{figure}[Shift registration of a dataset]{FIG:SHIFT}{Shift registration of a dataset}
  \subfigure[SBFIG:SHIFT1]{Unregistered curves $x(t)$}{\image{7cm}{}{0-empty}} \quad
  \subfigure[SBFIG:SHIFT2]{Registered curves $x^*(t)$}{\image{7cm}{}{0-empty}}
\end{figure}

A possible solution for the calculation of shifts $\delta_i$, is using a least
square criterion, defined as

$$
\text{REGSSE} = \sum_{i=1}{n}\int_{\mathcal{T}}\left [x_i(t+\delta_i) - \hat \mu(t) \right ]^2 dt
$$

The alignment problem will be based on finding the values $\delta_i$ that
minimize this criterion. For this purpose, we will use the derivatives of the
functions $x_i$, using a variation of the Newton-Rhapson method. A description
of the algorithm used to calculate these values is given in the Appendix B.1 [***********].
