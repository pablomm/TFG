In most cases, it is necessary to consider a more general transformation than a
simple translation; generally, this transformation will not be linear. In the
case of univariate functional data, i.e., the case that the functions under
analysis
depend on a single continuous parameter, we will have functions
$f_i: \mathcal{T} \subset \mathbb{R} \rightarrow \mathbb{R}$, in such a way that
we can understand
the problem as the search for an appropriate parameterization of our data,
according with the intrinsic structure of the dataset.

Consider the functions $\gamma_i: \mathcal{T} \rightarrow \mathcal{T}$,
referred to as warping functions in the related literature, that we will use to
reparametrize the domain, i.e, to change the internal scale of the data.
Using this warping functions we will
obtain the curves registered by means of composition of functions, i.e.,

\begin{equation}[]{Warping registration}
f_i^*(t)=f_i(\gamma_i(t)) = f_i \circ \gamma _i.
\end{equation}

So that the alignment does not alter the structure of the functional data,
these functions $\gamma_i$ must be boundary-preserving dipheomorphisms.
A dipheomorphisms is an invertible function that maps one differentiable
manifold to another such that both the function and its inverse are smooth.
In our case the dipheomorphisms maps the domain of the functions $f_i$,
$\mathcal{T}$, to itself. The boundary-preserving condition imposes that the
border of $\mathcal{T}$ is not altered by the warping functions.
In the
case where the domain of the functions $\mathcal{T}$ is an interval $[a,b]$, the
warpings will be strictly increasing functions that fix the bounds of the
domain, i.e., $\gamma_i(a)=a$ and $\gamma_i(b)=b$, as could be seen in the
figure \ref{FIG:WARPINGS}.

\begin{figure}[Set of warping functions]{FIG:WARPINGS}{Set of warping functions defined in $\mathcal{T}=[0,1]$}
  \includegraphics[width=8cm]{random-warpings}
\end{figure}

Without loss of generality, in the following sections, we
will assume that $\mathcal{T}=[0,1]$, because the general case
$\mathcal{T}=[a,b]$ can be reduced to this with an affine transformation. Also,
we will denote the set of such warping functions as $\Gamma$.
