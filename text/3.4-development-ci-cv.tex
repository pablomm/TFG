During the development of the package has been used an agile methodology, with
similar precepts than the eXtreme programming \cite{eXtreme}, which is based on
simplicity in development, continuous feedback and communication between
team members.

For this communication in conjunction with version control git has been using,
along with the functionalities available in Github. This platform offers a
series of tools for the review and control of contributions. There are multiple
workflows designed to carry out version control with git, of which we employ
Gitflow.

Gitflow is a widely used workflow, which uses git branches to organize the work.
It structures the code around two main branches: master and develop. The first
is a stable branch, where any built-in commit must be ready to go into
production and generates a new version. The second one, develop, is the code
that will make up the next planned version of the project. To add new code to
these branches, it is necessary to create other auxiliary ones that will be
merged to develop by means of a pull request, after a review process and pass
the bench tests. Figure \ref{FIG:GITFLOW} illustrates the mechanics of this
workflow.


\begin{figure}[Example of git flow branches]{FIG:GITFLOW}{Example of git flow branches}
  \image{8cm}{}{0-empty}
\end{figure}


One of the main advantages of using github, over other version control systems,
is the possibility of using a continuous integration mechanism. Travis CI has
been used for this purpose, which compiles the library and runs the bench tests
each time a commit is made, as in figure \ref{FIG:CHECKS}, where it may be seen a red cross
or a green tick associated to each commit depending on the bench tests. In
addition, it can also automate a wide variety of tasks, such as generating the
documentation, compiling or integrating style reviewers.

\begin{figure}[Example of travis checks]{FIG:CHECKS}{Example of travis checks}
  \image{12cm}{}{travis-checks}
\end{figure}

* Picture from https://es.atlassian.com/git, licensed under a Creative Commons Attribution 2.5 Australia License.
