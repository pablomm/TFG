
In spline interpolation, as in the linear case, we will define a piecewise
function, but employing polynomials to join the different points known.
The main advantage of using splines of order $n$, i.e., using polynomial of
order $n$, is that we will not only obtain continuous functions, as in the
linear case, but we will also obtain functions $\mathcal{C}^{n-1}[a,b]$.
This fact will be crucial in FDA, because we will need to use derivatives in
many phases of the analysis.

To achieve this, we will have to match the values of the derivatives of the
adjacent splines in the interpolation knots. If we denote by
$p_k(t)=\sum_{j=0}^n c_{jk} t^k$ to the spline defined in the region
$[t_{k-1}, t_{k}]$, during the calculation of the coefficients $c_{jk}$, we must
impose the restriction $p_{k}^(d)(t_k) = p_{k+1}^(d)$ for $d=1, \dots, n-1$.
For this purpose, we will define a linear system of equations which will be
solved iteratively. Figure \ref{FIG:SPLINE} shows the result of interpolate a
temporal series using splines of different orders (\ref{SBFIG:SPLINE1}), and the
derivation with these interpolators (\ref{SBFIG:SPLINE2}).

\begin{figure}{FIG:SPLINE}{Example of spline interpolation}
  \subfigure[SBFIG:SPLINE1]{Function interpolated}{\image{3cm}{}{0-empty}} \quad
  \subfigure[SBFIG:SPLINE2]{Splines derivatives $\partial_t$}{\image{3cm}{}{0-empty}}
\end

We can extend spline interpolation for multivariate functions, such as surfaces,
where bivariate splines will be used. In this case, we will use triangulation
to calculate the regions in which the polynomials will be defined, which will
be of the form $p(x, y) = \sum_{0 \le i + j \le n} \c_{i,j}x^i y^j$.
In the figure \ref{SBFIG:CUBICSPLINE1} it is shown the result of the
interpolation of a surface
using bicubic splines, and a partial derivative of the surface
\ref{SBFIG:CUBICSPLINE2}.

\begin{figure}{FIG:CUBICSPLINE}{Surface interpolated using bicubic splines}
  \subfigure[SBFIG:CUBICSPLINE1]{Surface interpolated}{\image{3cm}{}{0-empty}} \quad
  \subfigure[SBFIG:CUBICSPLINE2]{Partial derivative $\partial_x$}{\image{3cm}{}{0-empty}}
\end
