The \textit{representation} module includes general functionalities for the
representation of functional data and classes for processing this data using
object-oriented design.
There were two classes, \textit{FDataBasis}
and \textit{FDataGrid}, to represent the data in basis or discretized form, as it was explained in the indroduction of the chapter \ref{CAP:STATEOFART}.
These classes contain common functionalities.
In order to unify these functionalities, an abstract class called \textit{FData}
has been created, from which the previous ones inherit. This class implements methods for the
evaluation of the data as functions and its plotting.
Aside from methods for the composition of functions or operations between them.
The method created for the evaluation of functional data in \textit{FData}
allows its evaluation
as function vectors $f = (f_1, f_2, \dots, f_n)'$ where
$f_i:\mathbb{R}^d\rightarrow\mathbb{R}^m$, using a similar syntax to the mathematical
notation $f(t)$. It is also allowed to call these functions in a vectorized way with multiple values, so that

\begin{equation}[]{Evaluation of functional data}
f((t_1, t_2, \dots, t_k)) =
\begin{pmatrix}
    f_1((t_1, t_2, \dots, t_k))\\
    f_2((t_1, t_2, \dots, t_k))\\
    \vdots\\
    f_n((t_1, t_2, \dots, t_k))
  \end{pmatrix}
=
\begin{pmatrix}
f_1(t_1) & f_1(t_2) & \dots & f_1(t_k) \\
f_2(t_1) & \ddots &  & f_2(t_k) \\
\vdots &  & \ddots & \vdots \\
f_n(t_1) & f_n(t_2) & \dots & f_n(t_k) \\
\end{pmatrix}
%\left[\begin{array}{ccc}{f_1(t_1) & {f_2(t_2) & {\dots} \\ {\vdots} & {\ddots} & {} \\ {f_n(t_1) & {} & {a_{K K}}\end{array}\right,
\end{equation}

where $t_1, t_2, \dots t_k$ are the points of evaluation in $\mathbb{R}^d$.

For example, the code \ref{COD:EVAL} shows the creation of a set of three random observations
defined in $[0, 1]$. These random observations are packed in a \textit{FData} object
using a discrete representation. The samples are then evaluated at $0$, $0.5$ and $1$.

\lstset{frame=lines}
\lstset{caption={Example of evaluation of an \textit{FData}}}
\lstset{label={COD:EVAL}}
%\lstset{basicstyle=\footnotesize}

\begin{lstlisting}[language=Python]
>>> from skfda.datasets import make_sinusoidal_data
>>>
>>> f = make_sinusoidal_process(n_samples=3, start=0, stop=1)
>>> f([0, 0.5, 1]).round(3)
array([[-0.677,  0.368, -0.372],
       [ 0.702, -0.717,  0.765],
       [-0.8  ,  0.405, -0.703]])
\end{lstlisting}



The evaluation method separates the evaluation points $t_j$ into two sets, depending on
whether they are inside or outside the domain range of the functions. The points
within the domain range are passed to the evaluation kernel to its evaluation,
which will depend  on the implementation of the representation. In the discrete
representation it is used interpolation to this purpose.

A submodule called \textit{interpolation} has been created with different
interpolation methods, used in the class \textit{FDataGrid} as evaluation kernel.
In this module may be found the \textit{SplineInterpolation} class,
which implements the different interpolation techniques explained in the
section \ref{SEC:INTERPOLATION}.

In addition, \textit{FData} objects have an extrapolator, to which
points outside their domain are passed during the evaluation. It was created
a submodule called \textit{extrapolation} with different extrapolators used in
this task,
in which they have included the classes \textit{PeriodicExtrapolation}, to
extend the domain periodically, \textit{BoundExtrapolation}, to use the
values of the limits or \textit{FillExtrapolation}, to return a fixed value.

Once the data is in functional form using an \textit{FData} object,
it is possible to use the rest of the functionalities of the package, such as
the registration methods or the models for regression and classification.
