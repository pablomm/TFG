
We will define the pairwise alignment problem [REF TO Srivastava] to deal with
the registration problem of two functions using a global criterion.

Let $f_1, f_2 \in \mathcal{F}$ be functional observations and
$E: \mathcal{F} \times \mathcal{F} \rightarrow \mathbb{R}^+$ an energy
functional. The alignment problem may be understood as the search of a warping
function $\gamma^*$ which minimizes the energy between the two functions, i.e.,

$$
\gamma^* = argmin_{\gamma \in \Gamma} E[f_1, f_2 \circ \gamma] .
$$

When a warping $\gamma^*$ fulfills this property, we will say that $f_1$ is
registered to $f_2 \circ \gamma^*$. In the figure 2.11 it is shown an example in
 which the distance $\mathbb{L}^2$ has been used as energy, defined as
 $\|f - g\|_{\mathbb{L}^2}^2 = \int |f(t) - g(t)|^2 dt$.

To estimate $\gamma^*$, we will explore different paths that the
reparameterization may take in a discretized grid, trying to minimize the energy
term. In the Appendix B.2 it is described in detail the algorithm used for this
task, which is usually referred by several authors as dynamic programing
algorithm, or simply DPA [REF to DPA], because it makes use of this programming
technique to search the optimal path.

\begin{figure}[Pairwise alignment]{FIG:PAIRWISE}{Pairwise alignment}
  \subfigure[SBFIG:PAIRWISE1]{Alignment of $f_1$ to $f_2$}{\image{7cm}{}{0-empty}} \quad
  \subfigure[SBFIG:PAIRWISE2]{Warping built by the DPA algorithm}{\image{7cm}{}{0-empty}}
\end{figure}

Given a set of functions $\{f_i\}_{i=1}^n \subset \mathcal{F}$, the
groupwise alignment problem will consist in the search of warping functions
$\{\gamma_i^* \}_{i=1}^n \Subset \Gamma$ to align each of the functions with the
rest of them. To achieve this, we will build a target function $\mu$, also
called template, to which all the curves will be aligned. For instance,
the cross-sectional mean of the functions may be used as target.

To register each function we will search a warping $\gamma_i^*$ that minimizes
the energy, i.e.,

$$
\gamma_i^* = argmin_{\gamma \in \Gamma} E[\mu, f_i \circ \gamma],
$$

as well as we did in the pairwise alignment.

In practice, the metric $\mathbb{L}^2$ is not used as energy, because three
problems will arise that will not make it adequate: the pinching effect, the
lack of symmetry and the inverse inconsistency [REF to MAROON PAPER].
