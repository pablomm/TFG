
As mentioned above, the scikit-fda project started in 2017, at which time an
analysis of the requirements of the package was made \cite{FDA2018}. These
requirements are still in effect today, some of them are:

\begin{itemize}
\item The software developed has to be a Python package.
\item It has to be an open-source project.
\item The software must follow Python standards defined in PEP 8 and PEP 257.
\item Documentation has to be intended for a very general audience.
\item The project has to include and extensive test bench of unit test and continuous integration mechanism.
\end{itemize}
In addition to the original ones, three new requirements have been educed:

\begin{itemize}
\item The software should be cross-platform and the mechanism of
 continuous integration should run the test bench in the main operating systems, that is, 
 Linux, MacOs and Windows, as well as in the different Python versions supported.
 
\item  API should be similar, as far as possible, to the numpy[numpy citation], scipy[scipy citation] and
 scikit-learn[sklearn citation] ones, allowing whenever possible the use of their functionalities with the
  objects of the software developed.
\item The documentation should contain examples showing different functionalities.
\end{itemize}
