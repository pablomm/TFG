It is useful to quantify the amount of variability due to the phase and the
amplitude, in order to understand the data and validate the registration
procedure. For this purpose, in Kneip and Ramsay (2008)
\cite{RamsayAlois2008} it is developed an effective method for quantifying
this variation once the data have been aligned.

Let $\{x_i\}_{i=1}^N$ be a set of functional observations, and
$\{y_i\}_{i=1}^N = \{x_i \circ \gamma_i\}_{i=1}^N$ the corresponding registered
observations. Also, we will denote as $\bar x$ and $\bar y$ the
cross-sectional means. The Total Mean Square Error is defined as

\begin{equation}[]{Mean square error total}
\text{MSE}_{total}=  \frac{1}{N}\sum_{i=1}^{N} \|x_i(t)-\overline x(t)\|^2 =
\frac{1}{N}\sum_{i=1}^{N}\int[x_i(t)-\overline x(t)]^2dt .
\end{equation}

This error can be decomposed[??] as
$\text{MSE}_{total} = \text{MSE}_{amp} + \text{MSE}_{phase}$. This allow separate
the variability due to the phase and the amplitude.

\begin{equation}[]{Mean square decomposed}
\text{MSE}_{amp} =  C_R \frac{1}{N}
        \sum_{i=1}^{N} \int \left [ y_i(t) - \overline{y}(t) \right ]^2 dt \quad
\text{MSE}_{phase}=
        \int \left [C_R \overline{y}^2(t) - \overline{x}^2(t) \right]dt
\end{equation}

Where $C_R$ is a constant related with the covariance between $\dot \gamma_i$
and $y_i^2$.

%$$
%%C_R = 1 + \frac{\frac{1}{N}\sum_{i}^{N}\int [\dot \gamma(t)-\overline{\dot \gamma}(t)]
%        [ y_i^2(t)- \overline{y^2}(t) ]dt}
%        {\frac{1}{N} \sum_{i}^{N} \int y_i^2(t)dt}
%$$

From this decomposition it is possible to define the

square error multiple correlation index, which indicates the proportion of the
variation due to the phase
explained by the registration process. This index is defined as

\begin{equation}[]{Square multiple correlation index}
R^2 = \frac{\text{MSE}_{phase}}{\text{MSE}_{total}}.
\end{equation}

For instance, by quantifying the alignment of the dataset of the figure
\ref{FIG:SHIFT}, we get a value of \\ $R^2=0.99$, which indicates that the $99\%$ of
the variability it is produced by the phase.
