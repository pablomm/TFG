It will be useful to quantify the amount of variability due to the phase and the
amplitude, in order to understand the data and validate the registration
procedure. For this purpose, in Kneip and Ramsay (2008)
[Ref to Kneip and Ramsay] it is developed an effective method for quantifying
this variation once the data have been aligned.

Let $\{x_i\}_{i=1}^N$ be a set of functional observations, and
$\{y_i\}_{i=1}^N = \{x_i \circ \gamma_i\}_{i=1}^N$ the corresponding registered
observations. Also, we will denote as $\bar x$ and $\bar y$ the
cross-sectional means. The Total Mean Square Error is defined as

$$
\text{MSE}_{total}=  \frac{1}{N}\sum_{i=1}^{N} \|x_i(t)-\overline x(t)\|^2 =
\frac{1}{N}\sum_{i=1}^{N}\int[x_i(t)-\overline x(t)]^2dt .
$$

We can decompose the error in the part to each of the two sources of variability.

$$
\text{MSE}_{amp} =  C_R \frac{1}{N}
        \sum_{i=1}^{N} \int \left [ y_i(t) - \overline{y}(t) \right ]^2 dt \quad
\text{MSE}_{phase}=
        \int \left [C_R \overline{y}^2(t) - \overline{x}^2(t) \right]dt
$$

Where $C_R$ is a constant related with the covariance between $\dot \gamma_i$
and $y_i^2$. May be  proved[*] that $\text{MSE}_{total} = \text{MSE}_{amp} + \text{MSE}_{phase}$.

%$$
%%C_R = 1 + \frac{\frac{1}{N}\sum_{i}^{N}\int [\dot \gamma(t)-\overline{\dot \gamma}(t)]
%        [ y_i^2(t)- \overline{y^2}(t) ]dt}
%        {\frac{1}{N} \sum_{i}^{N} \int y_i^2(t)dt}
%$$

From this separation we will construct a square multiple correlation index
$R^2$, which will indicate the proportion of the variation due to the phase
explained by our registration process. This index will be defined as

$$
R^2 = \frac{\text{MSE}_{phase}}{\text{MSE}_{total}}.
$$

For instance, by quantifying the alignment of the dataset of the figure
\ref{FIG:SHIFT}, we get a value of $R^2=0.99$, which indicates that the $99\%$ of
the variability it is produced by the phase .
