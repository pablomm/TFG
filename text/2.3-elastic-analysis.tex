As discussed in previous chapter, the variability in functional data may come
from the continuous structure of its domain. This problem does not appear
exclusively in functional data analysis, other fields such as shape analysis or
computer vision should deal with this kind of variability.

In the classical approach, phase variation is separated in the registration of
the data, as a preprocessing step. However, in the elastic analysis approach the
separation of the two sources of variability is incorporated as a fundamental
part of the analysis. The term \textit{elastic} comes from the fact that we will allow
deformations of the domain of the functions throughout the analysis.

In Srivastava et. al. (2011) present in \cite{Srivastava2011} a novel
framework to treat this approach using the Fisher-Rao metric. Although this
framework covers a wide variety of topics, such as classification, regression or
functional component analysis\cite{Tucker2014}, in this chapter we will focus
on the part concerning the so-called elastic registration of functional data.
For that, we will follow the explanation outlined on chapters 1, 3, 4 and 8 of
the book Functional and Shape Data Analysis \cite{Srivastava2016} and
the implementation in the R-package \textit{fdasrvf}\cite{fdasrvf}.
