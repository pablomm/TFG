
In this chapter we will make a brief summary of the technologies and development
followed during this work, in which the mathematical concepts discussed
throughout the section \ref{SEC:STATEOFART} have been incorporated in
\textit{scikit-fda}. On the one hand, we will discuss the requirements and
design raised in development. During the design stage it has been special
emphasis on creating an API similar to the numpy and scipy ecosystem, allowing
the integration of functionalities of both libraries. This facilitates the use
of the package to developers and researchers who already know this ecosystem,
widely used for Statistics and Machine Learning in Python. Because of these
features, and to allow for wider diffusion of the package, it was decided to
develop the project as a texit{scikit}, which are open source packages for
scientific computing in Python.

\begin{figure}[scikit-fda logo]{FIG:SCIKITFDA}{scikit-fda logo}
	\image{10cm}{}{scikitfda2}
\end{figure}

On the other hand, it presents the technologies used for the development of
the package, which currently supports Python 3.6 and 3.7.
Among the tools used in this development has been used \textit{git},
to perform version control. This tool has allowed the communication of the team
involved in the development, and the integration of a continuous integration
system for the execution of tests and generation of online documentation.

In addition, there is a brief summary of the methodology followed in this work,
which had to be integrated together with the work of a team.
For this integration it has been necessary a great communication,
carrying out weekly meetings throughout the year. This communication has
been possible thanks to the version control system, which has allowed
revisions of the work among the members of the development team.
 Due to the open-source nature of the project, these revisions and the code
 are publicly available. For this reason special emphasis has been placed on
 the quality of the code, which will be maintained and used by multiple
 developers throughout the life of the project.
