%% Los margenes, tipo de hoja y estilo BOOK
\documentclass[a4paper,11pt,twoside,openright,titlepage]{book}
\usepackage[a4paper,left=1in,right=1in,top=0.6in]{geometry}

\usepackage[T1]{fontenc}    %Ineterprete de t�ldes
%\usepackage[Latin1]{inputenc}
\usepackage{amsmath,amssymb}    %Paquete de entornos matematicos

\usepackage{natbib}
\usepackage[english,spanish]{babel}
\selectlanguage{spanish} 
\usepackage{graphicx}
\usepackage{psfrag}
\usepackage{quotchap}
\usepackage{epsfig}
\usepackage[all]{xy}

\usepackage{epsfig}
\usepackage{makeidx}
\usepackage{ifthen}
\usepackage{multicolpar}    %Para poner texto en columnas en plan articulo intercalado con texto normal
\usepackage{multicol,multirow}

\usepackage{url}        %Para direcciones web
\usepackage{marvosym}   %Para imprimir el simbolo de \EUR euro
%\usepackage{eurosym}   %Para imprimir el simbolo de \euro euro
\usepackage{fancybox}   %Para tablas con bordes redondeados


%% Modificaci�n de la plantilla para adaptarla a los requisitos de PFC
\usepackage{fancyhdr}
\pagestyle{fancy}
%%% Cabeceras y pies de p�gina
\fancyhead[CE,CO]{\emph{\titulo}}
\fancyhead[LE,LO,RE,RO]{}
\fancyfoot[LE,RO]{\thepage}
\fancyfoot[CE,CO]{\leftmark}

\renewcommand{\footrulewidth}{.6pt}


%Definiciones de funciones para los titulos
\newlength\salto
\setlength{\salto}{3.5ex plus 1ex minus .2ex}

\newlength\resalto
\setlength{\resalto}{2.3ex plus.2ex}

\newcommand{\lsection}[1]
                {\section{#1}
                \vskip-.9\resalto   %%%% Aqu� reculo el posible salto por defecto de \section
                \hrule
                \vskip+.9\salto}  %%%% vuelvo ha realizar el salto (puedes poner otra vez el 90%)


%Para im�genes de entornos est�ticos \captionFigure{Texto Caption}{Texto Label}
\newcommand{\captionFigure}[2]{
    \refstepcounter{figure}
    \centerline{Figura \thefigure: #1 \label{#2}}
    \addcontentsline{lof}{section}{\thefigure.\ #1\label{#2}}
}

%Para im�genes de entornos est�ticos \NOcaptionFigure{Texto Caption}{Texto Label} "No escribe el caption"
\newcommand{\NOcaptionFigure}[2]{
    \refstepcounter{figure}
    \addcontentsline{lof}{figure}{\thefigure.\ #1\label{#2}}
}


%% Datos del PFC
\newcommand{\titulo}{T�tulo del TFG}
\newcommand{\autor}{Autor: Nombre Apellido1 Apellido2}
\newcommand{\director}{Nombre Apellido1 Apellido2}
\newcommand{\tutor}{Tutor: Nombre Apellido1 Apellido2}
\newcommand{\ponente}{Ponente: Nombre Apellido1 Apellido2}
\newcommand{\vocal}{Nombre Apellido1 Apellido2}
\newcommand{\vocalsup}{Nombre Apellido1 Apellido2}
\newcommand{\presidente}{Nombre Apellido1 Apellido2}
\newcommand{\presidentesup}{Nombre Apellido1 Apellido2}
\newcommand{\fecha}{MES 20xx}
\newcommand{\carrera}{Grado en ...}

\begin{document}
\setlength{\baselineskip}{18pt}  %% Espacio interlinea
\setlength{\parskip}{6pt plus 1pt minus 1pt} %% Espacio interp�rrafo

\begin{titlepage}

\begin{center}

\vspace*{2cm}

\LARGE \textsc{Universidad Aut�noma de Madrid}\\

\vspace{.2cm}

\large \textsc{Escuela polit�cnica superior}\\

\vspace{.2cm}

\begin{figure}[h]
    \begin{center}
        \begin{minipage}[c]{0.495\linewidth}
            \rightline{\epsfig{figure=images/logo_eps.eps,width=0.5\linewidth}}
        \end{minipage}
        \begin{minipage}[c]{0.495\linewidth}
            \leftline{\epsfig{figure=images/logo_uam.eps,width=0.5\linewidth}}
        \end{minipage}
    \end{center}
    \label{fig:Escudos}
\end{figure}

\Huge \carrera\\

\vspace{1cm}

\Huge \textsc{Trabajo Fin de Grado}\\

\vspace{1.5cm}

\Huge \MakeUppercase{\textbf{\titulo}}

\vspace{3cm}


\Large \autor\\
\Large \tutor\\
\Large \ponente\\

\vspace{0.5cm}

\Large \fecha

\end{center}

\end{titlepage}

\normalsize


\newpage \thispagestyle{empty} % P�gina vac�a


\frontmatter %Define el cuerpo inicial del libro en numeraci�n con letras romanas

\chapter*{}

\vspace*{0.2cm}

\begin{center}

\Huge \MakeUppercase{\textbf{\titulo}}

\vspace{7cm}

\Large \autor \\
\Large \tutor \\
\Large \ponente \\

\vspace{5cm}


Grupo de la EPS (opcional) \\
Dpto. de XXXXX \\
Escuela Polit�cnica Superior \\
Universidad Aut�noma de Madrid \\
\fecha

\end{center}

\normalsize

\newpage \thispagestyle{empty} % P�gina vac�a


\chapter*{Resumen}

\section*{Resumen}


\section*{Palabras Clave}

\newpage

%-------------------------------------------------------------------------------------------------------------------------------------
\section*{Abstract}


\section*{Key words}


\input{agradecimientos}

\tableofcontents

\newpage \thispagestyle{empty} % P�gina vac�a

\addcontentsline{toc}{chapter}{�ndice de Figuras}    %Para que aparezca en el �ndice
\renewcommand{\listfigurename}{�ndice de Figuras} 
\listoffigures

\newpage \thispagestyle{empty} % P�gina vac�a

\addcontentsline{toc}{chapter}{�ndice de Tablas}    %Para que aparezca en el �ndice
\renewcommand{\listtablename}{�ndice de Tablas} 
\listoftables

\newpage \thispagestyle{empty} % P�gina vac�a

\mainmatter %Define el cuerpo principal del libro numeraci�n normal.

% \input{preambulo}

\chapter{Introducci�n} 
\label{chap:intro}

\vspace{-0.2cm}

\lsection{Motivaci�n del proyecto}
Ejemplo de referencia a la bibliograf�a~\cite{article:Ejemplo}.

Ejemplo de imagen:
\begin{figure}[h]
  \centerline{
    \mbox{\includegraphics[width=3.00in]{images/logo_eps.eps}}
  }
  \caption{Ejemplo pie de figura 1}
  \label{fig:norm_Daugman}
\end{figure}

\lsection{Objetivos y enfoque}

\lsection{Metodolog�a y plan de trabajo}

Otro ejemplo de imagen:
\begin{figure}[h]
  \centerline{
    \mbox{\includegraphics[width=3.00in]{images/logo_uam.eps}}
  }
  \caption{Ejemplo pie de figura 2}
  \label{fig:norm_Daugman}
\end{figure}

\newpage \thispagestyle{empty} % P�gina vac�a 

\chapter{Reconocimiento de iris. Estado del arte}
\label{chap:estadodelarte}

\lsection{Introducci�n}

\lsection{Historia, nacimiento y evoluci�n.} \label{sec:historia}

\lsection{La anatom�a del ojo}
\label{sec:anatomiaojo}
\subsection{Aspectos diferenciadores del iris}


\lsection{Adquisici�n del Iris} \label{sec:adquisicion}
\subsection{Introducci�n}
\subsection{Esquemas de adquisici�n tradicionales}
\subsection{Consideraciones sobre la iluminaci�n}
\label{subsec:iluminacion}
\subsection{Posicionamiento del Iris}
\subsection{Sistemas comerciales de adquisici�n}


\lsection{Localizaci�n y segmentaci�n del Iris} \label{sec:localizacion}
\subsection{Introducci�n}
\subsection{Metodolog�a de J. Daugman y derivadas}
\subsection{Metodolog�a de R. Wildes y derivadas}
\subsection{Otras metodolog�as}
\subsection{Comparativa de metodolog�as}
\subsection{Detecci�n de pesta�as y ruido}


\lsection{Normalizaci�n del tama�o}
\label{sec:normalizacion}
\subsection{Daugman's Rubber Sheet Model}
\subsection{Image Registration}
\subsection{Normalizaci�n en �ngulo}
\subsection{Mejora del contraste y eliminaci�n de ruido}


\lsection{Algoritmos de Codificaci�n}
\label{sec:codificacion}
\subsection{Metodolog�a de Daugman: Filtros de Gabor}
\subsection{Metodolog�as alternativas a la de Daugman}
\subsubsection{Filtros Log-Gabor} \label{subsubsec:filtrosLogGabor}
\subsubsection{Wavelets}
\subsubsection{Haar Wavelet}
\subsubsection{Transformada Discreta del Coseno (DCT)}
\subsection{Metodolog�as de Wildes. Vectores de caracter�sticas reales (no binarios)}


\lsection{Algoritmos de Matching}
\label{sec:matching}
\subsection{Introducci�n}
\subsection{Distancia de Hamming} 
\label{subsec:distHamming}
\subsection{Distancia eucl�dea ponderada}
\subsection{Correlaci�n normalizada}


\lsection{Problem�tica y retos futuros}
\label{sec:problematica}
\subsection{Segmentaci�n}
\subsection{Captura ideal no invasiva}


\lsection{Competiciones o Evaluaciones de Iris}
\label{sec:competiciones}
\subsection{The Iris Challenge Evaluation (ICE)}
\subsection{The Noisy Iris Challenge Evaluation (NICE)}


\lsection{Bases de datos} \label{sec:databases}
\subsection{CASIA} \label{sec:CASIA_database}
\subsection{BioSec Baseline y BioSecurID} \label{sec:ATVS_database}



\newpage \thispagestyle{empty} % P�gina vac�a 

\chapter{Sistema, dise�o y desarrollo}
\label{chap:sistemadesarrollado}

\lsection{Segmentaci�n}

\lsection{Normalizaci�n}

\lsection{Codificaci�n}

\lsection{Matching}


\chapter{Experimentos Realizados y Resultados}
\label{chap:experimentos}

\lsection{Bases de datos y protocolo}

\lsection{Sistemas de referencia}
\label{sect:sistemasreferencia}

\lsection{Escenarios de pruebas}
\label{sect:escenarios_pruebas}
 
\lsection{Experimentos del sistema completo}


\chapter{Conclusiones y trabajo futuro}
\label{chap:conclusiones}


\newpage \thispagestyle{empty} % P�gina vac�a 

\chapter*{Glosario de acr�nimos}
\addcontentsline{toc}{chapter}{Glosario de acr�nimos}

\begin{itemize}
\item{\textbf{IS}:  Iris Subject}
\item{\textbf{DCT}: Discrete Cosine Transform}
\item{\textbf{WED}: Weighted Euclidean Distance}

\end{itemize}

\newpage \thispagestyle{empty} % P�gina vac�a

\addcontentsline{toc}{chapter}{Bibliograf�a}    %Agregamos al �ndice el capitulo de bibliograf�a 

\bibliographystyle{unsrt}   %plain pero ordenado en orden de aparacicion en documento principal
\bibliography{bibliografia}

\appendix   %Indicamos que lo que viene a continuaci�n son ap�ndices

%\frontmatter %Para poner los anexos en numeros romanos

\chapter{Manual de utilizaci�n}
\label{Anexo:manualuso}


\newpage \thispagestyle{empty} % P�gina vac�a 

\chapter{Manual del programador}
\label{Anexo:codigosMatlab}


\newpage \thispagestyle{empty} % P�gina vac�a 

%Hoja final en blanco
\newpage \thispagestyle{empty} % P�gina vac�a

\end{document}
