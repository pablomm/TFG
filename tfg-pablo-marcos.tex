
\documentclass[epsbased,copyleft,final,printable,covers,extendedindex,firstnumbered,tfg, english]{tfgtfmthesisuam}

\usepackage{amsmath}
\usepackage{amsfonts}
\usepackage{ mathrsfs }
\usepackage{tikz}
\usetikzlibrary{cd}


\author{Pablo Marcos Manchón}
\advisor{Alberto Suárez González}
\levelin{Mathematics and Computer Science}
\title[Functional data analysis: interpolation, registration and nearest neighbors in scikit-fda]{Functional data analysis in scikit-fda}
\subtitle{}


\copyrightdate{June 2019}

%\dedication{A mi mujer y a mis hijos}
\famouscite{Lo peor es cuando has terminado un capítulo\\y la máquina de escribir no aplaude. \\[0.1em] \begin{flushright}Orson Welles\end{flushright}}
%\prefacefile{inicio/prefacio}

\ackfile{text/0-acknowledgments}
\abstractfile{text/0-abstract}
\resumenfile{text/0-resumen}
\keywords{Functional data analysis, interpolation, registration,
nearest neighbors, python}

\palabrasclave{Análisis de datos funcionales,
interpolación, registro, vecinos próximos, python}


%\bibliographyconfig{tfgtfmthesisuam}

%\datadir{data}
\graphicsdir{img}
%\logosdir{img}
%\codesdir{codes}

\begin{document}

\chapter{Introduction\label{CAP:INTRODUCTION}}{text/1-introduction}
  \section{Motivation\label{SEC:MOTIVATION}}{text/1.1-motivation}
  \section{Goals and scope\label{SEC:GOALS}}{text/1.2-goals}
  \section{Document Structure\label{SEC:STRUCTURE}}{text/1.3-structure}


\chapter{State of the art\label{CAP:STATEOFART}}{text/2-state-of-the-art}

  \section{Functional data representation\label{SEC:REPRESENTATION}}{text/2.1-representation}
    \subsection{Interpolation\label{SEC:INTERPOLATION}}{text/2.1.1-interpolation}
      \subsubsection{Linear interpolation\label{SSEC:LINEAR}}{text/2.1.1.1-linear-interpolation}
      \subsubsection{Spline interpolation\label{SSEC:SPLINES}}{text/2.1.1.2-spline-interpolation}
      \subsubsection{Smoothing spline interpolation\label{SSEC:SSPLINES}}{text/2.1.1.3-smoothing-splines}

  \section{Registration\label{SEC:REGISTRATION}}{text/2.2-registration}
    \subsection{Shift registration\label{SEC:SHIFT}}{text/2.2.1-shift-registration}
    \subsection{Warping functions\label{SEC:WARPING}}{text/2.2.2-warping-functions}
    \subsection{Landmark registration\label{SEC:LANDMARK}}{text/2.2.3-landmark-registration}
    \subsection{Pairwise and groupwise alignment\label{SEC:L2PAIRWISE}}{text/2.2.4-pairwise-groupwise}
      \subsubsection{Pinching effect\label{SSEC:PINCHING}}{text/2.2.4.1-pinching-effect}
      \subsubsection{Lack of symmetry\label{SSEC:SIMMETRY}}{text/2.2.4.2-lack-of-symmetry}
    \subsection{Amplitude phase decomposition\label{SEC:DECOMPOSITION}}{text/2.2.5-decomposition}

    \section{Elastic methods in functional data analysis\label{SEC:ELASTIC}}{text/2.3-elastic-analysis}
      %\subsection{Fisher-Rao metric\label{SEC:FISHERRAO}}{text/2.3.1-fisher-rao}
      %\subsection{Amplitude and phase spaces\label{SEC:AMPPHA}}{text/2.3.2-amp-pha-spaces}
      %\subsection{Pairwise alignment\label{SEC:PAIRWISE}}{text/2.3.3-pairwise}
      %\subsection{Karcher means\label{SEC:KARCHER}}{text/2.3.4-karcher-mean}
      %\subsection{Elastic registration\label{SEC:ELASTICREG}}{text/2.3.5-elastic-registration}




\end{document}
