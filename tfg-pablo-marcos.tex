
\documentclass[epsbased,copyleft,final,loe, lof,extendedindex,firstnumbered,tfg,english]{tfgtfmthesisuam}

\usepackage{amsmath}
\usepackage{amsfonts}
\usepackage{mathrsfs}
\usepackage{bbm}
\usepackage{tikz}
\usetikzlibrary{cd}


\author{Pablo Marcos Manchón}
\advisor{Alberto Suárez González}
\levelin{Mathematics and Computer Science}
\title{Functional data analysis: interpolation, registration and nearest neighbors in scikit-fda}
%\subtitle{}


\copyrightdate{June 2019}

%\dedication{A mi mujer y a mis hijos}
\famouscite{Lo peor es cuando has terminado un capítulo\\y la máquina de escribir no aplaude. \\[0.1em] \begin{flushright}Orson Welles\end{flushright}}
%\prefacefile{inicio/prefacio}

\ackfile{text/0-acknowledgments}
\abstractfile{text/0-abstract}
\resumenfile{text/0-resumen}
\keywords{Functional data analysis, interpolation, registration,
nearest neighbors, python}

\palabrasclave{Análisis de datos funcionales,
interpolación, registro, vecinos próximos, python}


%\bibliographyconfig{tfgtfmthesisuam}

%\datadir{data}
\graphicsdir{img}
%\logosdir{img}
%\codesdir{codes}

\begin{document}

\chapter{Introduction\label{CAP:INTRODUCTION}}{text/1-introduction}
\section{Motivation\label{SEC:MOTIVATION}}{text/1.1-motivation}
\section{Goals and scope\label{SEC:GOALS}}{text/1.2-goals}
\section{Document Structure\label{SEC:STRUCTURE}}{text/1.3-structure}


\chapter{State of the art\label{CAP:STATEOFART}}{text/2-state-of-the-art}

\section{Functional data representation\label{SEC:REPRESENTATION}}{text/2.1-representation}
  \subsection{Interpolation\label{SEC:INTERPOLATION}}{text/2.1.1-interpolation}
    \subsubsection{Linear interpolation\label{SSEC:LINEAR}}{text/2.1.1.1-linear-interpolation}
    \subsubsection{Spline interpolation\label{SSEC:SPLINES}}{text/2.1.1.2-spline-interpolation}
    \subsubsection{Smoothing spline interpolation\label{SSEC:SSPLINES}}{text/2.1.1.3-smoothing-splines}

\section{Registration\label{SEC:REGISTRATION}}{text/2.2-registration}
  \subsection{Shift registration\label{SEC:SHIFT}}{text/2.2.1-shift-registration}
  \subsection{Warping functions\label{SEC:WARPING}}{text/2.2.2-warping-functions}
  \subsection{Landmark registration\label{SEC:LANDMARK}}{text/2.2.3-landmark-registration}
  \subsection{Pairwise and groupwise alignment\label{SEC:L2PAIRWISE}}{text/2.2.4-pairwise-groupwise}
    \subsubsection{Pinching effect\label{SSEC:PINCHING}}{text/2.2.4.1-pinching-effect}
    \subsubsection{Lack of symmetry\label{SSEC:SIMMETRY}}{text/2.2.4.2-lack-of-symmetry}
  \subsection{Amplitude phase decomposition\label{SEC:DECOMPOSITION}}{text/2.2.5-decomposition}

\section{Elastic methods in functional data analysis\label{SEC:ELASTIC}}{text/2.3-elastic-analysis}
  \subsection{Fisher-Rao metric\label{SEC:FISHERRAO}}{text/2.3.1-fisher-rao}
  \subsection{Amplitude and phase spaces\label{SEC:AMPPHA}}{text/2.3.2-amp-pha-spaces}
  \subsection{Pairwise alignment\label{SEC:PAIRWISE}}{text/2.3.3-pairwise}
  \subsection{Karcher means\label{SEC:KARCHER}}{text/2.3.4-karcher-mean}
  \subsection{Elastic registration\label{SEC:ELASTICREG}}{text/2.3.5-elastic-registration}
  \subsection{Restricting elasticity\label{SEC:RESTRICT}}{text/2.3.6-elasticity}

\section{Nearest neighbors\label{SEC:NEIGHBORS}}{text/2.4-nearest-neighbors}
  \subsection{Nearest neighbor search\label{SEC:SEARCH}}{text/2.4.1-nn-search}
  \subsection{Classification\label{SEC:SEARCH}}{text/2.4.2-classification}
  \subsection{Regression\label{SEC:SEARCH}}{text/2.4.3-regression}


\chapter{Design and development\label{CAP:DESIGNDEV}}{text/3-design-development}

\section{Analysis\label{SEC:ANALYSIS}}{text/3.1-analysis}
\section{Design\label{SEC:DESIGN}}{text/3.2-design}
  \subsection{Representation module\label{SEC:REPRMOD}}{text/3.2.1-representation}
  \subsection{Preprocessing module\label{SEC:PREPMOD}}{text/3.2.2-preprocessing}
  \subsection{Machine learning module\label{SEC:MLMOD}}{text/3.2.3-ml}
  \subsection{Miscellaneous module\label{SEC:MISCMOD}}{text/3.2.4-misc}
  \subsection{Dataset module\label{SEC:DATAMOD}}{text/3.2.5-dataset}
\section{Coding, documenting and testing\label{SEC:CODING}}{text/3.3-coding-docs-tests}
\section{Development, version control and continuous integration\label{SEC:DEVELOPMENT}}{text/3.4-development-ci-cv}

\chapter{Conclusions and future work\label{CAP:CONCLUSIONS}}{text/4-conclusions}

\appendix

\chapter{Algorithms\label{CAP:ALGORITHMS}}{text/A-algorithms}
  \section{Shift registration by the Newton-Raphson algorithm\label{SEC:NEWTON}}{text/A-algorithms}
  \section{Dynamic programming algorithm\label{SEC:DPA}}{text/A-algorithms}
  \section{Karcher mean in $\mathcal{F}$\label{SEC:KARCHERF}}{text/A-algorithms}
  \section{Karcher mean in $\Gamma$\label{SEC:KARCHERG}}{text/A-algorithms}
  \section{Karcher mean in $\mathscr{A}$\label{SEC:KARCHERA}}{text/A-algorithms}

\chapter{Proofs of mathematical results\label{CAP:PROOFS}}{text/B-proofs}



\chapter{Example notebooks\label{CAP:EXAMPLES}}{text/C-examples}
%Interpolation
%%C.2	 	Extrapolation
%C.3	 	Composition
%C.4	 	Shift registration
%C.5	 	Landmark shift
%C.6	 	Landmark registration
%C.7	 	Pairwise alignment
%C.8	 	Elastic registration
%C.9	 	K-nearest neighbors classification
%C.10	Radius Nearest neighbors classification
%C.11	Neighbors scalar regression
%C.12	Neighbors functional regression
%C.13	Nearest centroids
\chapter{Programmer’s guide\label{CAP:GUIDE}}{text/D-guide}

\end{document}
